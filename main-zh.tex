\documentclass[letterpaper,10pt]{ctexart}

\usepackage{latexsym}
\usepackage[empty]{fullpage}
\usepackage{titlesec}
\usepackage{marvosym}
\usepackage[usenames,dvipsnames]{color}
\usepackage{verbatim}
\usepackage{enumitem}
\usepackage[hidelinks]{hyperref}
\usepackage{fancyhdr}
\usepackage[english]{babel}
\usepackage{tabularx}
\usepackage{fontspec}

\setmainfont[Ligatures=TeX]{Linux Libertine O}
\setCJKmainfont{Noto Serif CJK SC}
\newCJKfontfamily[kai]\kaiti{AR PL KaitiM GB}

\pagestyle{fancy}
\fancyhf{} % clear all header and footer fields
\fancyfoot{}
\renewcommand{\headrulewidth}{0pt}
\renewcommand{\footrulewidth}{0pt}

% Adjust margins
\addtolength{\oddsidemargin}{-0.5in}
\addtolength{\evensidemargin}{-0.5in}
\addtolength{\textwidth}{1in}
\addtolength{\topmargin}{-.5in}
\addtolength{\textheight}{1.0in}

\urlstyle{same}

\raggedbottom
\raggedright
\setlength{\tabcolsep}{0in}

% Sections formatting
\titleformat{\section}{
  \vspace{-4pt}\scshape\raggedright\large
}{}{0em}{}[\color{black}\titlerule \vspace{-5pt}]

%-------------------------
% Custom commands
\newcommand{\resumeItem}[2]{
  \item\small{
    \textbf{#1}{: #2 \vspace{-2pt}}
  }
}

\newcommand{\resumeSubheading}[4]{
  \vspace{-1pt}\item
    \begin{tabular*}{0.97\textwidth}[t]{l@{\extracolsep{\fill}}r}
      \textbf{#1} & #2 \\
      \kaiti{\small#3} & \kaiti{\small #4} \\
    \end{tabular*}\vspace{-5pt}
}

\newcommand{\resumeSubSubheading}[2]{
    \begin{tabular*}{0.97\textwidth}{l@{\extracolsep{\fill}}r}
      \kaiti{\small#1} & \kaiti{\small #2} \\
    \end{tabular*}\vspace{-5pt}
}

\newcommand{\resumeSubItem}[2]{\resumeItem{#1}{#2}\vspace{-4pt}}

\renewcommand{\labelitemii}{$\circ$}

\newcommand{\resumeSubHeadingListStart}{\begin{itemize}[leftmargin=*]}
\newcommand{\resumeSubHeadingListEnd}{\end{itemize}}
\newcommand{\resumeItemListStart}{\begin{itemize}}
\newcommand{\resumeItemListEnd}{\end{itemize}\vspace{-5pt}}

\newcommand{\en}[1]{{\small{(#1)}}}

%-------------------------------------------
%%%%%%  CV STARTS HERE  %%%%%%%%%%%%%%%%%%%%%%%%%%%%


\begin{document}

%----------HEADING-----------------
\begin{tabular*}{\textwidth}{l@{\extracolsep{\fill}}r}
  \textbf{\Large 李烨锋} &
   \href{mailto:li3915@purdue.edu}{li3915@purdue.edu} \\
  & +852 56895775 或 +86 13238191225 \\
\end{tabular*}


%----------- Education -----------------
\section{教育}
  \resumeSubHeadingListStart
    \resumeSubheading
      {普渡大学\en{Purdue University}}{美国印第安纳州西拉法叶}
      {计算机科学哲学博士生}{即将入学}
    \resumeSubheading
      {香港科技大学}{香港特别行政区}
      {计算机科学与工程哲学硕士}{2018 年 9月 -- 2020 年 6月}
    \resumeSubheading
      {香港科技大学}{香港特别行政区}
      {计算机工程工程学学士;副修机器人学}{2014 年 9月 -- 2018 年 6月}
  \resumeSubHeadingListEnd
  
%----------- Research Experience -----------------
\section{研究}
\resumeSubHeadingListStart

  \resumeSubheading
    {香港科技大学计算机工程 VisGraph 实验室}{香港}
    {硕士生;导师为权龙教授}
    {2018 年 8月 -- 2020 年 5月}
    \resumeItemListStart
      \resumeItem{保存流形性质的单体收缩 \en{Manifoldness Preserving Contraction}(硕士论文)}{
        设计了一个新的单体\en{simplex}收缩方法来保证输入的流形复形\en{complex}在收缩后仍为流形。该方法使用了边缘补充以及奇点分裂的手段。这个收缩操作可以被应用在计算机几何处理上来取代时不时破坏拓扑结构的传统收缩操作。}
      \resumeItem{简化三角形三维网格}{
        利用了新设计的单体收缩操作来改善三角形三维网格简化的过程;利用了降低维度投影的办法来降低细节层次\en{LOD}平行运算难度以提升运算效率。}
    \resumeItemListEnd
    
  \resumeSubheading
    {香港科技大学}{香港}
    {本科四年级生;导师为彭邵邦\en{Brahim Bensaou}教授}
    {2017 年 9月 -- 2018 年 3月}
    \resumeItemListStart
      \resumeItem{中央化无线局域网\en{Centralized Wireless LANs}(本科论文)}{
        设计了一个中央化无线局域网的协议,目的是增加室内密集网络环境下的网络资源的利用效率。此协议建立在 CSMA/CA 的基础上,包含中央控制器和无线接入点新的行为模式。用 C 在 hostapd 源代码的基础上部分地实现了此协议并且利用开源的 OpenWrt Linux 系统进行了实验。}
    \resumeItemListEnd
    
  \resumeSubheading
    {香港科技大学计算机工程 VisGraph 实验室}{香港}
    {本科生研究计划参与者;导师为权龙教授}
    {2016 年 8月 -- 11月}
    \resumeItemListStart
      \resumeItem{无人机相片采集与应用}{
        研究了无人机相机和计算机视觉及图形的交叉应用,在安卓平台开发了实现操作手和无人机的实时互动的软件程序。}
    \resumeItemListEnd

\resumeSubHeadingListEnd
  
%----------- Industrial Experience -----------------
\section{行业}
\resumeSubHeadingListStart

 \resumeSubheading
   {珠科创新技术有限公司 \small{(现被苹果公司收购)}}{中国深圳与香港}
   {软件研发}
   {2019 年 6月 -- 8月}
   \resumeItemListStart
     \resumeItem{三角形网格处理}{
       将对三角形网格处理的收缩操作等研究成果转移至 Altizure,一个领先水平的基于云计算的三维重建应用和平台;使用 C++ 开发高效率的网格处理程序。}
         \resumeItem{ZRPC}{
       参与了 ZRPC 分布式运算远程调用系统的开发,开发语言是 Go。}
         \resumeItem{相片数据管理、校验和可视化}{
       使用 JavaScript 开发了一个完整的桌面应用来管理、校验和可视化大量(无人机捕捉的)相片数据。}
   \resumeItemListEnd
   
 \resumeSubheading
   {Dash Serviced Suites}{香港}
   {兼职 JavaScript 开发}
   {2018 年 2月 -- 5月}
   \resumeItemListStart
     \resumeItem{Web 开发}{
       在 DASH2 项目中参与网页前端、API 以及数据库的开发,此项目是该创业公司的一个市场网页应用。}
   \resumeItemListEnd

\resumeSubHeadingListEnd
  
%----------- Teaching Experience -----------------
\section{授课}
\resumeSubHeadingListStart

 \resumeSubheading
   {香港科技大学 COMP1021: Introduction to Computer Science}{香港}
   {助教}
   {2019 年 9月 -- 12月}

 \resumeSubheading
   {香港科技大学 COMP3311: Database Management Systems}{香港}
   {助教}
   {2018 年 9月 -- 12月}

\resumeSubHeadingListEnd
  
%----------- Teaching Experience -----------------
\section{课外}
\resumeSubHeadingListStart

 \resumeSubheading
   {计算机协会 Principles of Programming Languages 论坛}{美国路易斯安那州新奥尔良}
   {学生志愿者}
   {2020 年 1月}
   \resumeItemListStart
     \item 协助论坛活动的举办。
   \resumeItemListEnd

 \resumeSubheading
   {机甲大师赛}{中国深圳}
   {机械、计算机工程师}
   {2017 年 2月 -- 8月}
   \resumeItemListStart
     \item 与队友共同设计了香港科技大学参赛队伍在比赛中的主力机器人“英雄”的机械构造。英雄机器人可以由操作手远程操控来捕捉、发射弹丸,也可以用伸缩的腿部结构登岛。
   \resumeItemListEnd

 \resumeSubheading
   {香港科技大学中国民间艺术坊}{香港}
   {执行委员会 IT 秘书}
   {2015 年 2月 -- 2016 年 2月}
   \resumeItemListStart
     \item 独立制作社团网站;用 JavaScript 开发了“一站到底”学生竞赛活动所需的网页应用。
     \item 负责拍照和互联网线上宣传。
     \item 负责组织了长江三峡和香港龙脊的户外出游活动。
   \resumeItemListEnd

\resumeSubHeadingListEnd

%-------------------------------------------
\end{document}
