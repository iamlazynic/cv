\documentclass[11pt]{article}

\usepackage[letterpaper,total={7.2in,9.8in}]{geometry}
\usepackage{datetime2} % iso-format
\usepackage{fontspec}  % loading fonts (lualatex)
\usepackage{titlesec}  % customize \section
\usepackage{enumitem}  % customize \enum

% set fonts
%   requires: fontspec, installed fonts
\setmainfont{Linux Libertine O}
\setsansfont{Linux Biolinum O}
\newfontfamily\zhfont[Ligatures=TeX]{Noto Serif CJK SC}
% switch off page numbering
\pagenumbering{gobble}
% font size and rule for section headings
% spacing around subsection headings
%   requires: titlesec
\titleformat{\section}{\scshape\large}{\thesection}{1em}{}[{\titlerule[0.1pt]}]
\titleformat*{\subsection}{\normalsize\bfseries}
\titlespacing\subsection{0pt}{10pt plus 2pt minus 1pt}{0pt}
% do not indent paragraphs
\setlength{\parindent}{0pt}
% remove vertical spacing between items
%   requires: enumitem
\setlist{noitemsep}

% subsection with additional text floating right
\newcommand{\subsec}[2]{\subsection*{#1 \hfill {\normalfont\itshape #2}}}
% a description line in italic
\newcommand{\subsecdesc}[2]{{#1 \hfill \itshape #2}}

\title{
  \bfseries\underline{Li} Yefeng
  (\raisebox{-1pt}{\Large\zhfont 李烨锋})}
\author{
  \normalsize\sffamily li3915@purdue.edu \\
  \normalsize\sffamily +1 765-409-7247 \\
  \normalsize\sffamily https://www.cs.purdue.edu/homes/li3915}

\begin{document}

\maketitle

\section*{Research Interests} %-------------------------------------------------

  My research interests lie in the broad area of Programming Languages and
  Software Systems. I am interested in formal verification for distributed
  programming and for compilation, type systems, proof assistants, functional
  programming, and so on. I hope to develope rigorously founded theories and
  tools at the service of reliable software.

\section*{Education} %----------------------------------------------------------

  \subsec{Purdue University}
         {West Lafayette, IN, USA}
  \subsecdesc{Doctor of Philosophy student in Computer Science}
             {Jan. 2021--now}

  \subsec{Hong Kong University of Science and Technology (HKUST)}
         {Hong Kong S.A.R.}
  \subsecdesc{Master of Philosophy in Computer Science and Engineering}
             {Sept. 2018--June 2020} \\
  \subsecdesc{Bachelor of Engineering in Computer Engineering | Minor in Robotics}
             {Sept. 2014--June 2018}

\section*{Research Projects} %--------------------------------------------------

  \subsec{Manifoldness Preserving Contraction}
         {VisGraph Lab, CSE, HKUST}
  \subsecdesc{M.Phil. thesis, advised by Prof. Quan Long}
             {Aug. 2018--May 2020}
  \begin{itemize}
    \item Designed a new contraction method that guarantees manifold output
      given a manifold input using augmentation and the separation of
      singularities. It can be applied to geometry processing as a replacement
      to the conventional contraction operation which may destroy topology and
      produce imperfect results.
  \end{itemize}

  \subsec{Triangle mesh simplification}
         {VisGraph Lab, CSE, HKUST}
  \subsecdesc{Advised by Prof. Quan Long}
             {Aug. 2018--May 2020}
  \begin{itemize}
    \item Used new contraction techniques to improve triangle mesh
      simplification wherein traditional methods fail to effectively restrict
      the destruction of input topology.
    \item Facilitated the parallelization in the construction of
      Levels-of-Detail of 3D models using 2D-projection of tile boundaries.
  \end{itemize}

  \subsec{Centralized Wireless Local Area Networks}
         {HKUST}
  \subsecdesc{Undergraduate final-year thesis, advised by Prof. Brahim Bensaou}
             {Sept. 2017--Mar. 2018}
  \begin{itemize}
    \item Specified a centralization protocol for Wireless LANs as an
      (SDN) extension atop CSMA/CA to explore the
      improvement of resource utilization in dense indoor networks. It was
      partially implemented in C based on hostapd's source code and experimented
      on OpenWrt embedded Linux system.
  \end{itemize}

  \subsec{Photograph capturing with drones}
         {VisGraph Lab, CSE, HKUST}
  \subsecdesc{Undergraduate Research Opportunity Program, advised by Prof. Quan Long}
             {Aug.--Nov. 2016}
  \begin{itemize}
    \item Investigated in the application of Computer Vision and Graphics for
      Android devices as remote controls for drones, with a focus on human-drone
      interaction. An Android application was built to retrieve data from a
      geography database and visualize useful information on the screen
      according to the vision of the drone.
  \end{itemize}

\section*{Professional Experience} %--------------------------------------------

  \subsec{Everest Innovation Technology \small{(merged into Apple Inc.)}}
         {Shenzhen \& Hong Kong, China}
  \subsecdesc{Researcher \& Software Developer}
             {June--Aug. 2019}
  \begin{itemize}
    \item \textbf{Triangle mesh processing}: Transferred novel geometry
      processing techniques into \emph{Altizure}, a world-class cloud-based 3D
      reconstruction platform; developed efficient mesh processing program in
      C++.
    \item \textbf{ZRPC}: Participated in the development of \emph{ZRPC}, an RPC
      distributed computing framework, in Go.
    \item \textbf{Data management and visualization}: Developed a photographic
      data validation, management, and visualization desktop application in
      JavaScript.
  \end{itemize}

  \subsec{Dash Serviced Suites}
         {Hong Kong}
  \subsecdesc{Part-time JavaScript Developer}
             {Feb.--May 2018}
  \begin{itemize}
    \item \textbf{Web development}: Worked on the Web interface, API, and
      database management of \emph{DASH2}, an online marketplace Web application
      by the startup company.
  \end{itemize}

\section*{Teaching Experience} %------------------------------------------------

  \subsec{Department of Computer Science, Purdue University}
         {West Lafayette, IN, USA}
  \subsecdesc{Teaching Assistant}
             {2021--now}
  \begin{itemize}
    \item CS24000: Programming in C, Spring 2021
  \end{itemize}

  \subsec{Department of Computer Science and Engineering, HKUST}
         {Hong Kong}
  \subsecdesc{Teaching Assistant}
             {2018--2019}
  \begin{itemize}
    \item COMP1021: Introduction to Computer Science, Fall 2019
    \item COMP3311: Database Management Systems, Fall 2018
  \end{itemize}

\section*{Extracurricular Activities} %-----------------------------------------

  \subsec{ACM SIGPLAN Symposium on Principles of Programming Languages}
         {New Orleans, LA, USA}
  \subsecdesc{Student Volunteer}
             {Jan. 2020}
  \begin{itemize}
    \item Supported event organization.
  \end{itemize}

  \subsec{RoboMaster Robotics Competition}
         {Shenzhen, China}
  \subsecdesc{Mechanical/Computer Engineer}
             {Feb.--Aug. 2017}
  \begin{itemize}
    \item Co-designed the mechanical structure of \emph{Hero}, the main-force in
      this multi-robot contest, for RoboMaster HKUST team. Our Hero robot was
      controlled remotely, capable of capturing, storing and shooting bullets,
      and climbing onto stairs with telescopic legs.
  \end{itemize}

  \subsec{Chinese Folk-Art Society, HKUST}
         {Hong Kong}
  \subsecdesc{IT Secretary, Executive Committee}
             {Feb. 2015--Feb. 2016}
  \begin{itemize}
    \item Independently built the official website of our society and developed
      a Web application in JavaScript to assist the hosting of a knowledge
      competition named \emph{Who is Still Standing}.
    \item Took charge of photographing and Internet platform promotions.
    \item Organized trips to the Yangzi, China and Dragon's Back, Hong Kong.
  \end{itemize}

\end{document}
